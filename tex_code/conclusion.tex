\section{Conclusion}
	Due to the different bandwidths both techniques use, a direct comparison is not straightforward.
    We can, however, state that DSSS reaches the same performance in multiuser scenarios as our FHSS implementation while using a chipping rate of 16. This corresponds to a bandwidth of 32 Hz, where FHSS employed 1024 Hz.
    Our DSSS implementation also performed better in our "real-life noise" test (See ``\nameref{sec:real-life}'') and was more robust to Gaussian noise.
    FHSS performed better in our narrowband interference test scenario and is probably more flexible to scale up to more users.
    
    We could have rediscussed using FSK for fast-hopping in FHSS, as we think we could not properly show the potential of this technique with BPSK.
    
    The implementation of the jammer has been challenging and though we tried several different approaches, the result is not fully satisfying.
    A way to jam a certain band using a certain SNR or power density spectrum would certainly be practical and would possibly yield better results.
    But this was outside the time frame of this project and we also lack the background in signal processing to properly implement it.
    
    Further work could include support for simulation of free-space path loss, where the distance between the sender and the receiver would influence the quality of the signal.
    Additionally we could extend the free-space path loss by using a custom propagation coefficient to simulate more complex environments like buildings and other obstacles that prevent direct line of sight.
    
    Another project could be to not only measure bit-error rates, but also introduce a packet-error rate.
    To support this we would first need to decide how a packet should be defined.
    One possibility would be to combine a fixed amount of bits to a packet or send a special bit sequence to announce the beginning of a new packet.
    Additionally, some sort of error correction scheme would be necessary to discuss this approach.