\section{Experiment Setup}

	In the following sections we introduce the various test scenarios and the parameters that were used to obtain the final results.
	
	\subsection{Test Architecture}
	Our simulation of the spread spectrum mechanisms is based on a simple pipeline, where a \emph{sender} is writing its data onto a \emph{medium}. Various kinds of interference is generated by a \emph{jammer} which adds its data to the medium. Finally the \emph{receiver} reads data from the medium and tries to decode the original bit sequence.
	
	\paragraph{Sender}
	The sender is responsible to spread a random data sequence consisting of 0s and 1s using either the \emph{DSSS} or \emph{FHSS} spreading mechanism. After spreading, the sender will modulate the spread data onto passband using the BPSK modulation scheme with a carrier frequency $f_c$ in the case of DSSS. With FHSS, the sender generates a set of $n=8$ sub-carrier frequency bands of bandwidth $\Delta f$. The modulated signal is then passed to the \emph{medium}
	
	\paragraph{Medium} 
	The medium stores the spread data in frequency domain. After a signal is sent to the medium, we perform a Fast Fourier Transform (FFT) of the data. A jammer may add an interference signal onto the original data.
	
	\paragraph{Receiver}
	The receiver reads the 
	
	\paragraph{Jammer}
	
	
	\subsection{Testing Parameters}
	Throughout all test scenarios, the carrier frequency for all DSSS tests remained $f_c = 100\text{Hz}$. The channel configuration for FHSS is also constant, with the sub-carriers starting at frequency $f_c = 100\text{Hz}$ with a channel bandwidth of $\Delta f = 128 \text{Hz}$. The symbol rate was fixed at $f_{sym} = 8\text{Hz}$. In each test, white Gaussian noise with an SNR of $10\text{dB}$ was added to the time domain signal to simulate interference of various origins.
	
	
	\paragraph{Narrowband Interference}
	In this scenario, we tested the performance of the spreading schemes under the influence of narrowband interference. The parameters used are summarized in table \ref{tab:narrowband}
	
	
	\begin{center}
	    \begin{tabular}{ | l | l | l | p{5cm} |}
	    \hline
	    Day & Min Temp & Max Temp & Summary \\ \hline
	    Monday & 11C & 22C & A clear day with lots of sunshine.  
	    However, the strong breeze will bring down the temperatures. \\ \hline
	    Tuesday & 9C & 19C & Cloudy with rain, across many northern regions. Clear spells
	    across most of Scotland and Northern Ireland,
	    but rain reaching the far northwest. \\ \hline
	    Wednesday & 10C & 21C & Rain will still linger for the morning.
	    Conditions will improve by early afternoon and continue
	    throughout the evening. \\
	    \hline
	    \end{tabular}
	\end{center}
	
	
	\paragraph{Wideband Interference}
	\paragraph{Multiple Users}
	\paragraph{Varying Noise Bandwidth}
	\paragraph{Varying strength of white Gaussian noise}
	
	
	
	